\documentclass[blue]{NeptuneBall}
\begin{document}
\name{\bHuman{}}

The \cExExKing{} Ball presents interesting challenges for human guests, given that it is being held at King \cKing{\MYname{}}'s palace in \pAtlantis{}, several kilometers under the sea.\\

{\bf Breathing Under Water:}\\
The humans who are in attendance at this ball have had a special spell cast on them by \cManta{}, the \cKing{\King}'s most trusted advisor. \cManta{} is the only magician allowed to practice most magic in \pAtlantis{}. The spell allows humans to breathe underwater while within the grounds of the palace (all of game space is within the grounds of the palace). You will be escorted back to the surface in the morning. ({\bf You should consider game space closed - you may not leave.})\\

{\bf The Royal Family:}\\
The royal family of \pAmerica{} is small. \cEric{\King} \cEric{} rules \pAmerica{}, with \cEric{\their} \cAriel{\spouse}, \cAriel{\King} \cAriel{} at \cEric{\their} side. \cAriel{} is from \pAtlantis{}. Much of the country was shocked to learn that communication between the nations was even possible, never mind the idea that \pAtlantis{} might be friendly, given ongoing altercations with \pPacifica{}. \pAmerica{} has not yet fully accepted \cAriel{} as their \cAriel{\King}, despite \cAriel{\their} having lived among them for 14 years. On the other hand, the \cWillow{\offspring} of the royal couple, \cWillow{\Prince} \cWillow{}, is the pride and joy of the nation, adored by all.

The royal family has not always been so small. \cEric{}'s father was lost in a storm, 21 years ago. 7 years later, \cEric{}'s \cSlave{\sibling} was killed during an attack from the sea on the palace, leaving even more people suspicious of \cAriel{\King} \cAriel{}.\\

{\bf Religion:}\\
The human kingdom of \pAmerica{} practices a religion that worships the sun. The Sun God brings life to all things on the land. The summer solstice is the most sacred day of the year, and the midsummer festivals are elaborate and expensive. There is no greater omen than to be married under the noon-day sun on a clear day.

Humans fear the ocean, and its dark depths, for they conceal the mighty kraken. The kraken is hated by all and expeditions often set out to kill the beast. Only some of them return. It is believed that the storm that killed \cEric{}'s father was summoned by the kraken itself. Only a devil storm could have bested that crew and left no survivors.\\

{\bf Everyday life in \pAmerica{}:}\\
Life in \pAmerica{} is simple. \pAmerica{} is at peace with its neighbors, and many of its populace are farmers. A few brave souls are fishermen, or mariners, but for the most part, the citizens of \pAmerica{} prefer to stay on land and give thanks to the Sun God for his bounty.

Magic is extremely rare in \pAmerica{}. There is very little magic to be found anywhere on land. Magicians are therefore regarded with some suspicion. They are not necessarily outcasts though, and a few, like Merlin, have become prominent advisors to rulers.

\cPolio{} used to be a big problem in \pAmerica{}. It devastated much of the previous generation, and in some ways, \pAmerica{} is still rebuilding. Through the efforts of many dedicated scientists, and a particularly useful magician, a vaccine was developed. The epidemic has been brought under control and now most of the population has been vaccinated. \cPolio{} pops up occasionally in outlying villages, but for the most part, it is a thing of the past.\\ \\ \\

{\bf Relationship with the undersea kingdoms:}
\begin{itemize}
  \item {\bf \pAtlantis{}:} \pAmerica{} has slowly been building a tenuous relationship with \pAtlantis{}. Over the past couple of years, informal discussions  have begun to occur between the two nations, led by \cAriel{\King} \cAriel{}. Many people are hopeful that \pAtlantis{} could become a lucrative new trading partner.
  \item {\bf \pPacifica{}:} \pAmerica{} has a much less amicable relationship with \pPacifica{}. The merfolk of \pPacifica{} attack \pAmerica{}n ships on sight, and the concept of trade talks is laughable. No one really knows for sure why \pPacifica{} is so aggressive, but it may have something to do with the kraken.
\end{itemize}

\end{document}