\documentclass[green]{NeptuneBall}
\begin{document}
\name{\gTarot{}}

\updatemacro{\cWitch}{
  \nickname{Ursula}
  \mapnickinformal
}

You have the ability to use the Tarot to cast someone's fortune. You did not always have this ability -- it developed during your long captivity. You aren't sure whether \cWitch{} did something to you, or whether this is innate magic power that has bloomed under the ocean, where magic develops much more substantially than on land. Regardless of it's origin however, this ability is incredibly useful. It is certainly part of why \cWitch{} hasn't gotten rid of you yet.

The Tarot is a powerful aid for gathering information, but it is difficult to use without the target's permission. For castings that take less than a few hours to perform, you \emph{must} have the target's permission and cooperation. The quickest casting is a 3 card cast, and is your standard casting.\\

{\bf How to cast someone's Tarot:}
\begin{enumerate}
  \item Acquire verbal permission from the target to read their fortune.
  \item Find a quiet place where you wont be disturbed (or overheard), and settle on the floor or at a table with your target.
  \item A good caster always knows \cSlave{\their} target. Roleplay at least 1 minute of discussion with the target to learn about their past, and what they wish to learn about their future. At the end of this time, have the target tell you their ``Tarot'' Score. (The other two necessary cards are listed in the ``\mTarot{\MYname}'' notebook.)
  \item Cast the Tarot:
  \begin{enumerate}
    \item Verify which 2 other cards you need. (Refer to the triggers in the ``\mTarot{\MYname}'' notebook). Your goal is to get all 3 necessary cards in your ``Reading Hand.'' If at any point that you realize that the current distribution of cards makes this impossible, you may restart the mechanic from step (b).
    \item Shuffle the Tarot deck twice thoroughly (your deck includes ONLY the MAJOR arcana, 22 cards total).
    \item Deal out 7 cards. This is your ``Working Hand''. 
    \item Set aside {\bf 1} card from your ``Working Hand'' to keep. This is your ``Reading Hand''.
    \item Repeat steps (c) and (d) twice more, so that you have 3 cards total in your ``Reading Hand''. You will have 1 card left unrevealed.
    \item If you have succeeded in collecting the necessary cards, open the associated page of the ``\mTarot{\MYname}'' notebook and {\bf read out loud} what is written on the page. You must deliver this information honestly to your target.
    \item If you have not achieved the correct ``Reading Hand'', you \emph{may} start over from step (b). There is no penalty for failure, and no obligation to continue.
  \end{enumerate}
  \item Pick up the cards, the reading is now complete. You may wish to roleplay giving advice in an attempt to draw additional information out of your target, as the tarot is often cryptic.
\end{enumerate}

You cannot read your own Tarot in game because you already read it pre-game. The spirits were unusually helpful this time, and gave you some useful information about your readings at the ball:

\begin{itemize}
\item The people with the Devil in their draw are all current or former Assassins.
\item The people with the Magician in their draw are all current, former, or potential Magicians.
\item Those with The Lovers in their draw are in a secret relationship with someone.
\end{itemize}

\newpage
As a seer, you have an uncanny ability to observe someone and figure out what they desperately need to know.  You may observe a character for 30 seconds without talking to them, and then you may reveal to them what question the tarot can answer for them. Use this ability to encourage people to get their tarot read.

\begin{enumerate}
	\item Badge number \cPlant{\MYnumber} needs to know what danger the banquet holds.
	\item Badge number \cKing{\MYnumber} needs to know the immediate danger facing \cKing{\their} family.
	\item Badge number \cQueen{\MYnumber} needs to know the danger right under \cQueen{\their} nose.
	\item Badge number \cGeneral{\MYnumber} needs to know the immediate danger lurking for \cGeneral{\their} lover.
	\item Badge number \cWitch{\MYnumber} needs to know to safeguard her power.
	\item Badge number \cPriest{\MYnumber} needs to know how best to exact revenge.
	\item Badge number \cPrincess{\MYnumber} needs to know a secret about \cPrincess{\their} \cAthena{\parent}.
	\item Badge number \cAriel{\MYnumber} needs to know about the immediate dangers to \pAtlantis{}.
	\item Badge number \cWillow{\MYnumber} needs to know where the greatest opportunity for growth is.
	\item Badge number \cManta{\MYnumber} needs to know where the flaw in \cManta{\their} plan lies.
	\item Badge number \cBodyguard{\MYnumber} needs to know how to save \pPacifica{}.
	\item Badge number \cPrince{\MYnumber} needs to know why the treaty might fail.
	\item Badge number \cSpy{\MYnumber} needs to know how best to influence the treaty.
	\item Badge number \cDiplomat{\MYnumber} needs to know who not to trust.
\end{enumerate}
\end{document}